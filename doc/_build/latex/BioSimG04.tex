%% Generated by Sphinx.
\def\sphinxdocclass{report}
\documentclass[a4paper,10pt,english]{sphinxmanual}
\ifdefined\pdfpxdimen
   \let\sphinxpxdimen\pdfpxdimen\else\newdimen\sphinxpxdimen
\fi \sphinxpxdimen=.75bp\relax

\usepackage[utf8]{inputenc}
\ifdefined\DeclareUnicodeCharacter
 \ifdefined\DeclareUnicodeCharacterAsOptional\else
  \DeclareUnicodeCharacter{00A0}{\nobreakspace}
\fi\fi
\usepackage{cmap}
\usepackage[T1]{fontenc}
\usepackage{amsmath,amssymb,amstext}
\usepackage{babel}
\usepackage{times}
\usepackage[Bjarne]{fncychap}
\usepackage{longtable}
\usepackage{sphinx}

\usepackage{geometry}
\usepackage{multirow}
\usepackage{eqparbox}

% Include hyperref last.
\usepackage{hyperref}
% Fix anchor placement for figures with captions.
\usepackage{hypcap}% it must be loaded after hyperref.
% Set up styles of URL: it should be placed after hyperref.
\urlstyle{same}
\addto\captionsenglish{\renewcommand{\contentsname}{Contents:}}

\addto\captionsenglish{\renewcommand{\figurename}{Fig.}}
\addto\captionsenglish{\renewcommand{\tablename}{Table}}
\addto\captionsenglish{\renewcommand{\literalblockname}{Listing}}

\addto\extrasenglish{\def\pageautorefname{page}}

\setcounter{tocdepth}{1}



\title{BioSim G04 Documentation}
\date{Jan 22, 2018}
\release{2018.01.22}
\author{Lars Martin Lied, Jon-Fredrik Cappelen}
\newcommand{\sphinxlogo}{}
\renewcommand{\releasename}{Release}
\makeindex

\begin{document}

\maketitle
\sphinxtableofcontents
\phantomsection\label{\detokenize{index::doc}}

\begin{description}
\item[{This file is a simulation}] \leavevmode\begin{itemize}
\item {} 
for herbivores and carnivores

\item {} 
on an island

\end{itemize}

\end{description}


\chapter{Animals}
\label{\detokenize{animals:animals}}\label{\detokenize{animals:welcome-to-biosim-g04-s-documentation}}\label{\detokenize{animals::doc}}

\section{The animals module}
\label{\detokenize{animals:the-animals-module}}\phantomsection\label{\detokenize{animals:module-biosim.animals}}\index{biosim.animals (module)}\index{Carnivore (class in biosim.animals)}

\begin{fulllineitems}
\phantomsection\label{\detokenize{animals:biosim.animals.Carnivore}}\pysiglinewithargsret{\sphinxstrong{class }\sphinxcode{biosim.animals.}\sphinxbfcode{Carnivore}}{\emph{age}, \emph{weight}}{}
Constructor method for carnivore with entered age and weight.

The carnivore class is a subclass of the Herbivore class, using most of
the methods from Herbivore except the feeding method.
\begin{description}
\item[{age}] \leavevmode{[}int{]}
Age of animal

\item[{weight}] \leavevmode{[}float{]}
Weight of animal

\end{description}
\index{feeding() (biosim.animals.Carnivore method)}

\begin{fulllineitems}
\phantomsection\label{\detokenize{animals:biosim.animals.Carnivore.feeding}}\pysiglinewithargsret{\sphinxbfcode{feeding}}{\emph{preys}}{}
Feeding method for carnivores

Goes though all the herbivores in the same cell as the carnivore
and checks if the carnivore eats it or not. This depends on fitness of
both carnivore an herbivore, the parameters `DeltaPhiMax' and `beta',
and the appetite of the carnivore.
It also updates the weight and fitness of the carnivore.
\begin{description}
\item[{preys}] \leavevmode{[}list{]}
A list of all the herbivores on the same tile as the given carnivore

\end{description}
\begin{description}
\item[{eaten\_bool}] \leavevmode{[}list{]}
A list of true/false values for which herbivores to be eaten.
False means it is to be eaten, and removed for the cell.

\end{description}

\end{fulllineitems}


\end{fulllineitems}

\index{Herbivore (class in biosim.animals)}

\begin{fulllineitems}
\phantomsection\label{\detokenize{animals:biosim.animals.Herbivore}}\pysiglinewithargsret{\sphinxstrong{class }\sphinxcode{biosim.animals.}\sphinxbfcode{Herbivore}}{\emph{age}, \emph{weight}}{}
Constructor-method for making an herbivore of set age and weight.
\begin{description}
\item[{age}] \leavevmode{[}int{]}
Age of herbivore

\item[{weight}] \leavevmode{[}float{]}
Weight of herbivore

\end{description}
\index{aging() (biosim.animals.Herbivore method)}

\begin{fulllineitems}
\phantomsection\label{\detokenize{animals:biosim.animals.Herbivore.aging}}\pysiglinewithargsret{\sphinxbfcode{aging}}{}{}
Method for handling the aging of an animal. Also updates fitness.

\end{fulllineitems}

\index{death() (biosim.animals.Herbivore method)}

\begin{fulllineitems}
\phantomsection\label{\detokenize{animals:biosim.animals.Herbivore.death}}\pysiglinewithargsret{\sphinxbfcode{death}}{}{}
Method for for checking if the animal should die from natural causes

Calculates the chance of dying due to low fitness, by using the
parameter `omega'

\end{fulllineitems}

\index{feeding() (biosim.animals.Herbivore method)}

\begin{fulllineitems}
\phantomsection\label{\detokenize{animals:biosim.animals.Herbivore.feeding}}\pysiglinewithargsret{\sphinxbfcode{feeding}}{\emph{landscape\_instance}}{}
Handles the feeding of  the animal

landscape\_instance : The tile, that the given animal is in

\end{fulllineitems}

\index{get\_weight() (biosim.animals.Herbivore method)}

\begin{fulllineitems}
\phantomsection\label{\detokenize{animals:biosim.animals.Herbivore.get_weight}}\pysiglinewithargsret{\sphinxbfcode{get\_weight}}{}{}~\begin{description}
\item[{float}] \leavevmode
The weight of the animal

\end{description}

\end{fulllineitems}

\index{loss\_of\_weight() (biosim.animals.Herbivore method)}

\begin{fulllineitems}
\phantomsection\label{\detokenize{animals:biosim.animals.Herbivore.loss_of_weight}}\pysiglinewithargsret{\sphinxbfcode{loss\_of\_weight}}{}{}
Method for handling the loss of weight, by natural causes

A method for decreasing the weight of animal every year by a parameter
`eta' multiplied by the animals own weight. Also updates fitness.

\end{fulllineitems}

\index{migration() (biosim.animals.Herbivore method)}

\begin{fulllineitems}
\phantomsection\label{\detokenize{animals:biosim.animals.Herbivore.migration}}\pysiglinewithargsret{\sphinxbfcode{migration}}{}{}
Method for checking if the animal will migrate
\begin{description}
\item[{bool}] \leavevmode
True or False value; if the animal wil move or not

\end{description}

\end{fulllineitems}

\index{procreation() (biosim.animals.Herbivore method)}

\begin{fulllineitems}
\phantomsection\label{\detokenize{animals:biosim.animals.Herbivore.procreation}}\pysiglinewithargsret{\sphinxbfcode{procreation}}{\emph{landscape\_instance}, \emph{number\_of\_adults}}{}
Handles the procreation of the animals
\begin{description}
\item[{landscape\_instance}] \leavevmode{[}object{]}
The tile of the animal

\item[{number\_of\_adults}] \leavevmode{[}int{]}
Number of animals old enough to procreate

\end{description}

\end{fulllineitems}

\index{set\_parameters() (biosim.animals.Herbivore class method)}

\begin{fulllineitems}
\phantomsection\label{\detokenize{animals:biosim.animals.Herbivore.set_parameters}}\pysiglinewithargsret{\sphinxstrong{classmethod }\sphinxbfcode{set\_parameters}}{\emph{parameter\_changes}}{}
Method that allows the user to set parameter values for the animal.

Method that allows the user to set parameter values for the animal.
This replaces the default values.
\begin{description}
\item[{parameter\_changes}] \leavevmode{[}dictionary{]}
A dictionary with one or more keys to set new parameters for the
animal. The items should be numeric.

\end{description}
\begin{description}
\item[{KeyError}] \leavevmode
If parameter\_changes contain key(s) not already in the parameters.

\item[{ValueError}] \leavevmode
If parameter values that are to be set are not valid, such as
negative amounts of fodder/weight.

\end{description}

\end{fulllineitems}

\index{update\_fitness() (biosim.animals.Herbivore method)}

\begin{fulllineitems}
\phantomsection\label{\detokenize{animals:biosim.animals.Herbivore.update_fitness}}\pysiglinewithargsret{\sphinxbfcode{update\_fitness}}{}{}
Method for updating fitness.

The attribute self.fitness is updated.

\end{fulllineitems}


\end{fulllineitems}



\chapter{Island}
\label{\detokenize{island:island}}\label{\detokenize{island::doc}}

\section{The island module}
\label{\detokenize{island:the-island-module}}\phantomsection\label{\detokenize{island:module-biosim.island}}\index{biosim.island (module)}\index{Island (class in biosim.island)}

\begin{fulllineitems}
\phantomsection\label{\detokenize{island:biosim.island.Island}}\pysiglinewithargsret{\sphinxstrong{class }\sphinxcode{biosim.island.}\sphinxbfcode{Island}}{\emph{island\_map}}{}
Class for the island

Init method for class Island
\begin{description}
\item[{island\_map}] \leavevmode{[}str{]}
A string representing the map

\end{description}
\index{add\_animal\_island() (biosim.island.Island method)}

\begin{fulllineitems}
\phantomsection\label{\detokenize{island:biosim.island.Island.add_animal_island}}\pysiglinewithargsret{\sphinxbfcode{add\_animal\_island}}{\emph{coordinates}, \emph{animals\_list}}{}
Method for adding animals to the island
\begin{description}
\item[{coordinates}] \leavevmode{[}tuple{]}
Where to put the animals on the island

\item[{animals\_list}] \leavevmode{[}list{]}
The list of animals to put on the island

\end{description}
\begin{description}
\item[{ValueError}] \leavevmode
If coordinates is not in jungle, savannah or desert

\end{description}

\end{fulllineitems}

\index{animals\_on\_island() (biosim.island.Island method)}

\begin{fulllineitems}
\phantomsection\label{\detokenize{island:biosim.island.Island.animals_on_island}}\pysiglinewithargsret{\sphinxbfcode{animals\_on\_island}}{}{}
Method for getting the number of carnivores and herbivores on island

Makes one matrix for herbivores and one for carnivores, containing
the number of carnivores and herbivores on the tile on the island
matching the matrix.

\end{fulllineitems}

\index{array\_to\_island() (biosim.island.Island method)}

\begin{fulllineitems}
\phantomsection\label{\detokenize{island:biosim.island.Island.array_to_island}}\pysiglinewithargsret{\sphinxbfcode{array\_to\_island}}{}{}
Converts the array from the method string\_to\_array into a map

changes the array consisting of strings into an array consisting of
instances of classes depedning on the letter that was previously in
the array, for example `J' would turn into an instance of the
Jungle-class. These are stored in Island.cells

\end{fulllineitems}

\index{cell\_move\_carnivores() (biosim.island.Island method)}

\begin{fulllineitems}
\phantomsection\label{\detokenize{island:biosim.island.Island.cell_move_carnivores}}\pysiglinewithargsret{\sphinxbfcode{cell\_move\_carnivores}}{\emph{coordinate}}{}
method for moving herbivores in given cell, if they should move
\begin{description}
\item[{coordinate}] \leavevmode{[}tuple{]}
coordinates in which to move herbivores

\end{description}

\end{fulllineitems}

\index{cell\_move\_herbivores() (biosim.island.Island method)}

\begin{fulllineitems}
\phantomsection\label{\detokenize{island:biosim.island.Island.cell_move_herbivores}}\pysiglinewithargsret{\sphinxbfcode{cell\_move\_herbivores}}{\emph{coordinate}}{}
method for moving herbivores in given cell, if they should move
\begin{description}
\item[{coordinate}] \leavevmode{[}tuple{]}
coordinates in which to move herbivores

\end{description}

\end{fulllineitems}

\index{cycle() (biosim.island.Island method)}

\begin{fulllineitems}
\phantomsection\label{\detokenize{island:biosim.island.Island.cycle}}\pysiglinewithargsret{\sphinxbfcode{cycle}}{}{}
Simulates one cycle/year

Every method is done in the order specified in the assignment
Returns
-------

\end{fulllineitems}

\index{get\_direction() (biosim.island.Island static method)}

\begin{fulllineitems}
\phantomsection\label{\detokenize{island:biosim.island.Island.get_direction}}\pysiglinewithargsret{\sphinxstrong{static }\sphinxbfcode{get\_direction}}{\emph{pi\_values}}{}
Method for getting the direction for the animal to move in
\begin{description}
\item[{pi\_values}] \leavevmode{[}tuple{]}
Values used to calculate the direction, found in the method;
get\_pi\_values\_herbivores and get\_pi\_values\_carnivores

\end{description}

Direction as a string for example `right' or `left

\end{fulllineitems}

\index{get\_pi\_values\_carnivores() (biosim.island.Island method)}

\begin{fulllineitems}
\phantomsection\label{\detokenize{island:biosim.island.Island.get_pi_values_carnivores}}\pysiglinewithargsret{\sphinxbfcode{get\_pi\_values\_carnivores}}{\emph{coordinate}}{}
Method for calculating pi-values(propensity) for carnivores
\begin{description}
\item[{coordinate}] \leavevmode{[}tuple{]}
Coordinates  from which to find the animals to calculate pi's for

\end{description}
\begin{description}
\item[{pi\_right, pi\_up, pi\_left, pi\_down}] \leavevmode{[}float{]}
Different propensity-values for each direction in given cell

\end{description}

\end{fulllineitems}

\index{get\_pi\_values\_herbivores() (biosim.island.Island method)}

\begin{fulllineitems}
\phantomsection\label{\detokenize{island:biosim.island.Island.get_pi_values_herbivores}}\pysiglinewithargsret{\sphinxbfcode{get\_pi\_values\_herbivores}}{\emph{coordinate}}{}
Method for calculating pi-values(propensity) for herbivores
\begin{description}
\item[{coordinate}] \leavevmode{[}tuple{]}
Coordinates  from which to find the animals to calculate pi's for

\end{description}
\begin{description}
\item[{pi\_right, pi\_up, pi\_left, pi\_down}] \leavevmode{[}float{]}
Different propensity-values for each direction in given cell

\end{description}

\end{fulllineitems}

\index{get\_random\_coordinates() (biosim.island.Island method)}

\begin{fulllineitems}
\phantomsection\label{\detokenize{island:biosim.island.Island.get_random_coordinates}}\pysiglinewithargsret{\sphinxbfcode{get\_random\_coordinates}}{}{}
Method for getting random coordinates from the island

Shuffles the coordinates on the island, for use in a different method
excludes the ocean at the edge of the island
\begin{description}
\item[{points\_on\_island}] \leavevmode{[}list{]}
a randomly shuffled list of coordinates from the island

\end{description}

\end{fulllineitems}

\index{migration() (biosim.island.Island method)}

\begin{fulllineitems}
\phantomsection\label{\detokenize{island:biosim.island.Island.migration}}\pysiglinewithargsret{\sphinxbfcode{migration}}{}{}
Method for handling migration

Implements the previous move-methods for herbivores and carnivores,
including all new animals in that were appended in the move-methods

\end{fulllineitems}

\index{number\_of\_carnivores\_island() (biosim.island.Island method)}

\begin{fulllineitems}
\phantomsection\label{\detokenize{island:biosim.island.Island.number_of_carnivores_island}}\pysiglinewithargsret{\sphinxbfcode{number\_of\_carnivores\_island}}{}{}
Method for getting the total amount of herbivores on island

Number of carnivores on island

\end{fulllineitems}

\index{number\_of\_herbivores\_island() (biosim.island.Island method)}

\begin{fulllineitems}
\phantomsection\label{\detokenize{island:biosim.island.Island.number_of_herbivores_island}}\pysiglinewithargsret{\sphinxbfcode{number\_of\_herbivores\_island}}{}{}
Method for getting the total amount of herbivores on island

Number of herbivores on island

\end{fulllineitems}

\index{string\_to\_array() (biosim.island.Island method)}

\begin{fulllineitems}
\phantomsection\label{\detokenize{island:biosim.island.Island.string_to_array}}\pysiglinewithargsret{\sphinxbfcode{string\_to\_array}}{}{}
A method for concerting thr given string into an array
\begin{description}
\item[{map\_arr}] \leavevmode{[}arr{]}
An array of the map, where each letter has a seperate place

\end{description}

\end{fulllineitems}


\end{fulllineitems}



\chapter{Landscape}
\label{\detokenize{landscape::doc}}\label{\detokenize{landscape:landscape}}

\section{The landscape module}
\label{\detokenize{landscape:the-landscape-module}}\phantomsection\label{\detokenize{landscape:module-biosim.landscape}}\index{biosim.landscape (module)}\index{Desert (class in biosim.landscape)}

\begin{fulllineitems}
\phantomsection\label{\detokenize{landscape:biosim.landscape.Desert}}\pysigline{\sphinxstrong{class }\sphinxcode{biosim.landscape.}\sphinxbfcode{Desert}}
Desert is a subclass of Jungle, inheriting most of its methods
\index{feeding() (biosim.landscape.Desert method)}

\begin{fulllineitems}
\phantomsection\label{\detokenize{landscape:biosim.landscape.Desert.feeding}}\pysiglinewithargsret{\sphinxbfcode{feeding}}{}{}
Replaces the method Jungle.feeding. It does nothing

\end{fulllineitems}

\index{reset\_fodder() (biosim.landscape.Desert method)}

\begin{fulllineitems}
\phantomsection\label{\detokenize{landscape:biosim.landscape.Desert.reset_fodder}}\pysiglinewithargsret{\sphinxbfcode{reset\_fodder}}{}{}
Replaces the method Jungle.reset\_fodder. It does nothing

\end{fulllineitems}


\end{fulllineitems}

\index{Jungle (class in biosim.landscape)}

\begin{fulllineitems}
\phantomsection\label{\detokenize{landscape:biosim.landscape.Jungle}}\pysigline{\sphinxstrong{class }\sphinxcode{biosim.landscape.}\sphinxbfcode{Jungle}}
Class for the jungle, containing all methods used in the Jungle-landscape

Constructor for the jungle class
\index{add\_carnivore() (biosim.landscape.Jungle method)}

\begin{fulllineitems}
\phantomsection\label{\detokenize{landscape:biosim.landscape.Jungle.add_carnivore}}\pysiglinewithargsret{\sphinxbfcode{add\_carnivore}}{\emph{age}, \emph{weight}}{}
Method for adding a carnivore into the landscape-cell
\begin{description}
\item[{age}] \leavevmode{[}int{]}
The age for the carnivore to be added

\item[{weight}] \leavevmode{[}float{]}
The weight for the carnivore to be added

\end{description}

\end{fulllineitems}

\index{add\_herbivore() (biosim.landscape.Jungle method)}

\begin{fulllineitems}
\phantomsection\label{\detokenize{landscape:biosim.landscape.Jungle.add_herbivore}}\pysiglinewithargsret{\sphinxbfcode{add\_herbivore}}{\emph{age}, \emph{weight}}{}
Method for adding a herbivore into the landscape instance.
\begin{description}
\item[{age}] \leavevmode{[}int{]}
The age for the herbivore to be added

\item[{weight}] \leavevmode{[}float{]}
The weight for the animal to be added

\end{description}

\end{fulllineitems}

\index{aging() (biosim.landscape.Jungle method)}

\begin{fulllineitems}
\phantomsection\label{\detokenize{landscape:biosim.landscape.Jungle.aging}}\pysiglinewithargsret{\sphinxbfcode{aging}}{}{}
Method that makes all animals in the cell age

\end{fulllineitems}

\index{death() (biosim.landscape.Jungle method)}

\begin{fulllineitems}
\phantomsection\label{\detokenize{landscape:biosim.landscape.Jungle.death}}\pysiglinewithargsret{\sphinxbfcode{death}}{}{}
Method that makes some animals in the cell die, and remove them.

\end{fulllineitems}

\index{feeding() (biosim.landscape.Jungle method)}

\begin{fulllineitems}
\phantomsection\label{\detokenize{landscape:biosim.landscape.Jungle.feeding}}\pysiglinewithargsret{\sphinxbfcode{feeding}}{}{}
Method that makes all animals in the cell feed.

\end{fulllineitems}

\index{get\_abundance\_carnivore() (biosim.landscape.Jungle method)}

\begin{fulllineitems}
\phantomsection\label{\detokenize{landscape:biosim.landscape.Jungle.get_abundance_carnivore}}\pysiglinewithargsret{\sphinxbfcode{get\_abundance\_carnivore}}{}{}
Method for calculating abundance of food for carnivore in landscape.
\begin{description}
\item[{ek}] \leavevmode{[}float{]}
Abundance of food for carnivores in landscape

\end{description}

\end{fulllineitems}

\index{get\_abundance\_herbivore() (biosim.landscape.Jungle method)}

\begin{fulllineitems}
\phantomsection\label{\detokenize{landscape:biosim.landscape.Jungle.get_abundance_herbivore}}\pysiglinewithargsret{\sphinxbfcode{get\_abundance\_herbivore}}{}{}
Method for calculating abundance of food for herbivore in landscape.
\begin{description}
\item[{ek}] \leavevmode{[}float{]}
Abundance of food for herbivores in landscape.

\end{description}

\end{fulllineitems}

\index{get\_carnivores() (biosim.landscape.Jungle method)}

\begin{fulllineitems}
\phantomsection\label{\detokenize{landscape:biosim.landscape.Jungle.get_carnivores}}\pysiglinewithargsret{\sphinxbfcode{get\_carnivores}}{}{}
Method for getting a list of all carnivores in the landscape instance
\begin{description}
\item[{self.carnivores}] \leavevmode{[}list{]}
List of carnivores in landscape instance.

\end{description}

\end{fulllineitems}

\index{get\_fodder() (biosim.landscape.Jungle method)}

\begin{fulllineitems}
\phantomsection\label{\detokenize{landscape:biosim.landscape.Jungle.get_fodder}}\pysiglinewithargsret{\sphinxbfcode{get\_fodder}}{}{}
Method for getting fodder in the landscape instance.
\begin{description}
\item[{self.fodder}] \leavevmode{[}int{]}
Amount of fodder left is landscape instance.

\end{description}

\end{fulllineitems}

\index{get\_herbivores() (biosim.landscape.Jungle method)}

\begin{fulllineitems}
\phantomsection\label{\detokenize{landscape:biosim.landscape.Jungle.get_herbivores}}\pysiglinewithargsret{\sphinxbfcode{get\_herbivores}}{}{}
Method for getting a list of all herbivore in the landscape instance
\begin{description}
\item[{self.hebivores}] \leavevmode{[}list{]}
List of herbivores in instance

\end{description}

\end{fulllineitems}

\index{loss\_of\_weight() (biosim.landscape.Jungle method)}

\begin{fulllineitems}
\phantomsection\label{\detokenize{landscape:biosim.landscape.Jungle.loss_of_weight}}\pysiglinewithargsret{\sphinxbfcode{loss\_of\_weight}}{}{}
Method that makes all animals in the cell lose weight

\end{fulllineitems}

\index{migration() (biosim.landscape.Jungle method)}

\begin{fulllineitems}
\phantomsection\label{\detokenize{landscape:biosim.landscape.Jungle.migration}}\pysiglinewithargsret{\sphinxbfcode{migration}}{}{}
Dummy. Migration is handled by the class Island.

\end{fulllineitems}

\index{move\_new\_animals() (biosim.landscape.Jungle method)}

\begin{fulllineitems}
\phantomsection\label{\detokenize{landscape:biosim.landscape.Jungle.move_new_animals}}\pysiglinewithargsret{\sphinxbfcode{move\_new\_animals}}{}{}
Method for moving animals from the new-lists into the real one.

\end{fulllineitems}

\index{procreation() (biosim.landscape.Jungle method)}

\begin{fulllineitems}
\phantomsection\label{\detokenize{landscape:biosim.landscape.Jungle.procreation}}\pysiglinewithargsret{\sphinxbfcode{procreation}}{}{}
Method that makes all animals in the cell try to procreate

\end{fulllineitems}

\index{reduce\_fodder() (biosim.landscape.Jungle method)}

\begin{fulllineitems}
\phantomsection\label{\detokenize{landscape:biosim.landscape.Jungle.reduce_fodder}}\pysiglinewithargsret{\sphinxbfcode{reduce\_fodder}}{\emph{amount}}{}
Method for reducing the amount of fodder available in landscape instance
\begin{description}
\item[{amount}] \leavevmode{[}float, int{]}
How much fodder is to be removed from landscape instance.

\end{description}

\end{fulllineitems}

\index{reset\_fodder() (biosim.landscape.Jungle method)}

\begin{fulllineitems}
\phantomsection\label{\detokenize{landscape:biosim.landscape.Jungle.reset_fodder}}\pysiglinewithargsret{\sphinxbfcode{reset\_fodder}}{}{}
Method that resets the amount of fodder in the jungle to f\_max.

\end{fulllineitems}

\index{set\_parameters() (biosim.landscape.Jungle class method)}

\begin{fulllineitems}
\phantomsection\label{\detokenize{landscape:biosim.landscape.Jungle.set_parameters}}\pysiglinewithargsret{\sphinxstrong{classmethod }\sphinxbfcode{set\_parameters}}{\emph{parameter\_changes}}{}
Method that allows the user to set parameter values for the landscape.
\begin{description}
\item[{parameter\_changes}] \leavevmode{[}dict{]}
The changes to be made to the parameters

\end{description}

\end{fulllineitems}


\end{fulllineitems}

\index{Mountain (class in biosim.landscape)}

\begin{fulllineitems}
\phantomsection\label{\detokenize{landscape:biosim.landscape.Mountain}}\pysigline{\sphinxstrong{class }\sphinxcode{biosim.landscape.}\sphinxbfcode{Mountain}}
Class for mountain landscape. Is passive.

\end{fulllineitems}

\index{Ocean (class in biosim.landscape)}

\begin{fulllineitems}
\phantomsection\label{\detokenize{landscape:biosim.landscape.Ocean}}\pysigline{\sphinxstrong{class }\sphinxcode{biosim.landscape.}\sphinxbfcode{Ocean}}
Class for Ocean landscape. Is passive.

\end{fulllineitems}

\index{Savannah (class in biosim.landscape)}

\begin{fulllineitems}
\phantomsection\label{\detokenize{landscape:biosim.landscape.Savannah}}\pysigline{\sphinxstrong{class }\sphinxcode{biosim.landscape.}\sphinxbfcode{Savannah}}
Savannah is a subclass of Jungle, inheriting most of its methods
\index{reset\_fodder() (biosim.landscape.Savannah method)}

\begin{fulllineitems}
\phantomsection\label{\detokenize{landscape:biosim.landscape.Savannah.reset_fodder}}\pysiglinewithargsret{\sphinxbfcode{reset\_fodder}}{}{}
Method that updates the fodder amount in the savannah each year

\end{fulllineitems}


\end{fulllineitems}



\chapter{Simulation}
\label{\detokenize{simulation:simulation}}\label{\detokenize{simulation::doc}}

\section{The simulation module}
\label{\detokenize{simulation:the-simulation-module}}\phantomsection\label{\detokenize{simulation:module-biosim.simulation}}\index{biosim.simulation (module)}\index{BioSim (class in biosim.simulation)}

\begin{fulllineitems}
\phantomsection\label{\detokenize{simulation:biosim.simulation.BioSim}}\pysiglinewithargsret{\sphinxstrong{class }\sphinxcode{biosim.simulation.}\sphinxbfcode{BioSim}}{\emph{island\_map}, \emph{ini\_pop=None}, \emph{seed=12345}}{}
class for the simulation
\index{add\_population() (biosim.simulation.BioSim method)}

\begin{fulllineitems}
\phantomsection\label{\detokenize{simulation:biosim.simulation.BioSim.add_population}}\pysiglinewithargsret{\sphinxbfcode{add\_population}}{\emph{population}}{}
method for adding a population to the island
\begin{description}
\item[{population}] \leavevmode{[}list{]}
A list of animals, containing dictionaries for each animal

\end{description}

\end{fulllineitems}

\index{make\_carnivore\_density\_map() (biosim.simulation.BioSim method)}

\begin{fulllineitems}
\phantomsection\label{\detokenize{simulation:biosim.simulation.BioSim.make_carnivore_density_map}}\pysiglinewithargsret{\sphinxbfcode{make\_carnivore\_density\_map}}{}{}
Method for making the carnivore density map

\end{fulllineitems}

\index{make\_herbivore\_density\_map() (biosim.simulation.BioSim method)}

\begin{fulllineitems}
\phantomsection\label{\detokenize{simulation:biosim.simulation.BioSim.make_herbivore_density_map}}\pysiglinewithargsret{\sphinxbfcode{make\_herbivore\_density\_map}}{}{}
Method for making the hebrivore density map

\end{fulllineitems}

\index{make\_line\_plot() (biosim.simulation.BioSim method)}

\begin{fulllineitems}
\phantomsection\label{\detokenize{simulation:biosim.simulation.BioSim.make_line_plot}}\pysiglinewithargsret{\sphinxbfcode{make\_line\_plot}}{\emph{vis\_steps}}{}
Method for making the line plot

vis\_steps

\end{fulllineitems}

\index{make\_rgb\_map() (biosim.simulation.BioSim method)}

\begin{fulllineitems}
\phantomsection\label{\detokenize{simulation:biosim.simulation.BioSim.make_rgb_map}}\pysiglinewithargsret{\sphinxbfcode{make\_rgb\_map}}{}{}
Function to make RGB map from island-string

\end{fulllineitems}

\index{make\_visualization() (biosim.simulation.BioSim method)}

\begin{fulllineitems}
\phantomsection\label{\detokenize{simulation:biosim.simulation.BioSim.make_visualization}}\pysiglinewithargsret{\sphinxbfcode{make\_visualization}}{\emph{vis\_steps}}{}
Method for making the visualization
\begin{description}
\item[{vis\_steps}] \leavevmode{[}int{]}
How often the graphics should be updated, in years

\end{description}

\end{fulllineitems}

\index{reset\_axis\_limits() (biosim.simulation.BioSim method)}

\begin{fulllineitems}
\phantomsection\label{\detokenize{simulation:biosim.simulation.BioSim.reset_axis_limits}}\pysiglinewithargsret{\sphinxbfcode{reset\_axis\_limits}}{}{}
Method for resetting the axis limits for both x and y axis

\end{fulllineitems}

\index{reset\_color\_code\_limits() (biosim.simulation.BioSim method)}

\begin{fulllineitems}
\phantomsection\label{\detokenize{simulation:biosim.simulation.BioSim.reset_color_code_limits}}\pysiglinewithargsret{\sphinxbfcode{reset\_color\_code\_limits}}{}{}
Method for resetting the color code to default values

\end{fulllineitems}

\index{set\_axis\_limits() (biosim.simulation.BioSim method)}

\begin{fulllineitems}
\phantomsection\label{\detokenize{simulation:biosim.simulation.BioSim.set_axis_limits}}\pysiglinewithargsret{\sphinxbfcode{set\_axis\_limits}}{\emph{x\_limits}, \emph{y\_limits}}{}
Method for setting the x and y-limits on the line graph
\begin{description}
\item[{x\_limits}] \leavevmode{[}tuple, list{]}
tuple or list containing the lower and upper boundaries for the
x axis

\item[{y\_limits}] \leavevmode{[}tuple, list{]}
tuple or list containing the lower and upper boundaries for the
y axis

\end{description}

\end{fulllineitems}

\index{set\_color\_code\_limits() (biosim.simulation.BioSim method)}

\begin{fulllineitems}
\phantomsection\label{\detokenize{simulation:biosim.simulation.BioSim.set_color_code_limits}}\pysiglinewithargsret{\sphinxbfcode{set\_color\_code\_limits}}{\emph{herbivore\_colors}, \emph{carnivore\_colors}}{}
Method for setting the color code limits for each species
\begin{description}
\item[{herbivore\_colors}] \leavevmode{[}tuple{]}
boundaries for the color representation for number of herbivores

\item[{carnivore\_colors}] \leavevmode{[}tuple{]}
boundaries for the color representation for number of herbivores

\end{description}

\end{fulllineitems}

\index{simulate() (biosim.simulation.BioSim method)}

\begin{fulllineitems}
\phantomsection\label{\detokenize{simulation:biosim.simulation.BioSim.simulate}}\pysiglinewithargsret{\sphinxbfcode{simulate}}{\emph{num\_steps}, \emph{vis\_steps=1}, \emph{img\_steps=None}}{}
Method for simulating the entire island
\begin{description}
\item[{num\_steps}] \leavevmode{[}int{]}
Numbers of years to be simulated

\item[{vis\_steps}] \leavevmode{[}int{]}
How often the graphics should be updated, in years

\end{description}

img\_steps : How often the graphic should be saved

\end{fulllineitems}

\index{status\_number\_of\_animals\_by\_species() (biosim.simulation.BioSim method)}

\begin{fulllineitems}
\phantomsection\label{\detokenize{simulation:biosim.simulation.BioSim.status_number_of_animals_by_species}}\pysiglinewithargsret{\sphinxbfcode{status\_number\_of\_animals\_by\_species}}{}{}
method for getting number of herbivores and carnivores on screen
\begin{description}
\item[{dictionary}] \leavevmode{[}dict{]}
A dictionary containing number herbivores and carnivores on island

\end{description}

\end{fulllineitems}

\index{status\_number\_of\_animals\_total() (biosim.simulation.BioSim method)}

\begin{fulllineitems}
\phantomsection\label{\detokenize{simulation:biosim.simulation.BioSim.status_number_of_animals_total}}\pysiglinewithargsret{\sphinxbfcode{status\_number\_of\_animals\_total}}{}{}
method for getting the total amount of animals on screen

\end{fulllineitems}

\index{status\_per\_cell\_animal\_count() (biosim.simulation.BioSim method)}

\begin{fulllineitems}
\phantomsection\label{\detokenize{simulation:biosim.simulation.BioSim.status_per_cell_animal_count}}\pysiglinewithargsret{\sphinxbfcode{status\_per\_cell\_animal\_count}}{}{}
Method for getting number of each species in each cell, using pandas
\begin{description}
\item[{df}] \leavevmode{[}pandas.DataFrame{]}
A table containing each cell and the number of each species on
each cell

\end{description}

\end{fulllineitems}

\index{status\_year() (biosim.simulation.BioSim method)}

\begin{fulllineitems}
\phantomsection\label{\detokenize{simulation:biosim.simulation.BioSim.status_year}}\pysiglinewithargsret{\sphinxbfcode{status\_year}}{}{}
method for getting the current year on screen
\begin{description}
\item[{self.year}] \leavevmode{[}int{]}
the current year

\end{description}

\end{fulllineitems}

\index{update\_carnivore\_density\_map() (biosim.simulation.BioSim method)}

\begin{fulllineitems}
\phantomsection\label{\detokenize{simulation:biosim.simulation.BioSim.update_carnivore_density_map}}\pysiglinewithargsret{\sphinxbfcode{update\_carnivore\_density\_map}}{}{}
Method for updating the carnivore density map

\end{fulllineitems}

\index{update\_herbivore\_density\_map() (biosim.simulation.BioSim method)}

\begin{fulllineitems}
\phantomsection\label{\detokenize{simulation:biosim.simulation.BioSim.update_herbivore_density_map}}\pysiglinewithargsret{\sphinxbfcode{update\_herbivore\_density\_map}}{}{}
Method for updating the herbivore density map

\end{fulllineitems}

\index{update\_line\_plot() (biosim.simulation.BioSim method)}

\begin{fulllineitems}
\phantomsection\label{\detokenize{simulation:biosim.simulation.BioSim.update_line_plot}}\pysiglinewithargsret{\sphinxbfcode{update\_line\_plot}}{}{}
Method for updating the line plot

\end{fulllineitems}

\index{update\_visualization() (biosim.simulation.BioSim method)}

\begin{fulllineitems}
\phantomsection\label{\detokenize{simulation:biosim.simulation.BioSim.update_visualization}}\pysiglinewithargsret{\sphinxbfcode{update\_visualization}}{}{}
Method for updating the visualization

\end{fulllineitems}

\index{year\_counter() (biosim.simulation.BioSim method)}

\begin{fulllineitems}
\phantomsection\label{\detokenize{simulation:biosim.simulation.BioSim.year_counter}}\pysiglinewithargsret{\sphinxbfcode{year\_counter}}{}{}
Method for updating the counter on screen

\end{fulllineitems}


\end{fulllineitems}



\chapter{Tests}
\label{\detokenize{tests:tests}}\label{\detokenize{tests::doc}}

\section{The animal test module}
\label{\detokenize{tests:the-animal-test-module}}\phantomsection\label{\detokenize{tests:module-biosim.tests.test_animals}}\index{biosim.tests.test\_animals (module)}\index{TestAnimal (class in biosim.tests.test\_animals)}

\begin{fulllineitems}
\phantomsection\label{\detokenize{tests:biosim.tests.test_animals.TestAnimal}}\pysigline{\sphinxstrong{class }\sphinxcode{biosim.tests.test\_animals.}\sphinxbfcode{TestAnimal}}
class for testing animal
\index{set\_default\_animal() (biosim.tests.test\_animals.TestAnimal method)}

\begin{fulllineitems}
\phantomsection\label{\detokenize{tests:biosim.tests.test_animals.TestAnimal.set_default_animal}}\pysiglinewithargsret{\sphinxbfcode{set\_default\_animal}}{}{}
method for resetting to default parameters for herbivore and carnivore
Returns
-------

\end{fulllineitems}

\index{test\_aging() (biosim.tests.test\_animals.TestAnimal method)}

\begin{fulllineitems}
\phantomsection\label{\detokenize{tests:biosim.tests.test_animals.TestAnimal.test_aging}}\pysiglinewithargsret{\sphinxbfcode{test\_aging}}{}{}
Tests that the animal's age increases properly, including fitness
Returns
-------

\end{fulllineitems}

\index{test\_feeding\_carnivore\_appetite() (biosim.tests.test\_animals.TestAnimal method)}

\begin{fulllineitems}
\phantomsection\label{\detokenize{tests:biosim.tests.test_animals.TestAnimal.test_feeding_carnivore_appetite}}\pysiglinewithargsret{\sphinxbfcode{test\_feeding\_carnivore\_appetite}}{}{}
Test for a fit carnivore's feeding method, with low appetite
Returns
-------

\end{fulllineitems}

\index{test\_feeding\_carnivore\_fit() (biosim.tests.test\_animals.TestAnimal method)}

\begin{fulllineitems}
\phantomsection\label{\detokenize{tests:biosim.tests.test_animals.TestAnimal.test_feeding_carnivore_fit}}\pysiglinewithargsret{\sphinxbfcode{test\_feeding\_carnivore\_fit}}{}{}
Test for carnivore feeding method, with fit carnivore
Returns
-------

\end{fulllineitems}

\index{test\_feeding\_carnivore\_unfit() (biosim.tests.test\_animals.TestAnimal method)}

\begin{fulllineitems}
\phantomsection\label{\detokenize{tests:biosim.tests.test_animals.TestAnimal.test_feeding_carnivore_unfit}}\pysiglinewithargsret{\sphinxbfcode{test\_feeding\_carnivore\_unfit}}{}{}
Test for carnivore feeding method, with unfit carnivore
Returns
-------

\end{fulllineitems}

\index{test\_feeding\_little() (biosim.tests.test\_animals.TestAnimal method)}

\begin{fulllineitems}
\phantomsection\label{\detokenize{tests:biosim.tests.test_animals.TestAnimal.test_feeding_little}}\pysiglinewithargsret{\sphinxbfcode{test\_feeding\_little}}{}{}
Test for herbivore feeding method with little fodder
Returns
-------

\end{fulllineitems}

\index{test\_feeding\_none() (biosim.tests.test\_animals.TestAnimal method)}

\begin{fulllineitems}
\phantomsection\label{\detokenize{tests:biosim.tests.test_animals.TestAnimal.test_feeding_none}}\pysiglinewithargsret{\sphinxbfcode{test\_feeding\_none}}{}{}
Test for herbivore feeding method with no fodder
Returns
-------

\end{fulllineitems}

\index{test\_feeding\_plenty() (biosim.tests.test\_animals.TestAnimal method)}

\begin{fulllineitems}
\phantomsection\label{\detokenize{tests:biosim.tests.test_animals.TestAnimal.test_feeding_plenty}}\pysiglinewithargsret{\sphinxbfcode{test\_feeding\_plenty}}{}{}
Test for herbivore feeding method with plenty of fodder
Returns
-------

\end{fulllineitems}

\index{test\_init\_parameters() (biosim.tests.test\_animals.TestAnimal method)}

\begin{fulllineitems}
\phantomsection\label{\detokenize{tests:biosim.tests.test_animals.TestAnimal.test_init_parameters}}\pysiglinewithargsret{\sphinxbfcode{test\_init\_parameters}}{}{}
test for init\_parameters, that we can change parameters
Returns
-------

\end{fulllineitems}

\index{test\_loss\_of\_weight() (biosim.tests.test\_animals.TestAnimal method)}

\begin{fulllineitems}
\phantomsection\label{\detokenize{tests:biosim.tests.test_animals.TestAnimal.test_loss_of_weight}}\pysiglinewithargsret{\sphinxbfcode{test\_loss\_of\_weight}}{}{}
Tests that the animal loses weight, including fitness update
Returns
-------

\end{fulllineitems}

\index{test\_set\_parameters() (biosim.tests.test\_animals.TestAnimal method)}

\begin{fulllineitems}
\phantomsection\label{\detokenize{tests:biosim.tests.test_animals.TestAnimal.test_set_parameters}}\pysiglinewithargsret{\sphinxbfcode{test\_set\_parameters}}{}{}
test for method set\_parameters
Returns
-------

\end{fulllineitems}


\end{fulllineitems}



\section{The island test module}
\label{\detokenize{tests:the-island-test-module}}\phantomsection\label{\detokenize{tests:module-biosim.tests.test_island}}\index{biosim.tests.test\_island (module)}\index{TestIsland (class in biosim.tests.test\_island)}

\begin{fulllineitems}
\phantomsection\label{\detokenize{tests:biosim.tests.test_island.TestIsland}}\pysigline{\sphinxstrong{class }\sphinxcode{biosim.tests.test\_island.}\sphinxbfcode{TestIsland}}
class for testing island
\index{set\_default\_animal() (biosim.tests.test\_island.TestIsland method)}

\begin{fulllineitems}
\phantomsection\label{\detokenize{tests:biosim.tests.test_island.TestIsland.set_default_animal}}\pysiglinewithargsret{\sphinxbfcode{set\_default\_animal}}{}{}
method for resetting parameters for herbivores and carnivores
Returns
-------

\end{fulllineitems}

\index{test\_add\_animal\_island() (biosim.tests.test\_island.TestIsland method)}

\begin{fulllineitems}
\phantomsection\label{\detokenize{tests:biosim.tests.test_island.TestIsland.test_add_animal_island}}\pysiglinewithargsret{\sphinxbfcode{test\_add\_animal\_island}}{}{}
test for adding animals to map
Returns
-------

\end{fulllineitems}

\index{test\_array\_to\_island() (biosim.tests.test\_island.TestIsland method)}

\begin{fulllineitems}
\phantomsection\label{\detokenize{tests:biosim.tests.test_island.TestIsland.test_array_to_island}}\pysiglinewithargsret{\sphinxbfcode{test\_array\_to\_island}}{}{}
test for converting the array into a map
Returns
-------

\end{fulllineitems}

\index{test\_cell\_move\_herbivore\_and\_carnivore() (biosim.tests.test\_island.TestIsland method)}

\begin{fulllineitems}
\phantomsection\label{\detokenize{tests:biosim.tests.test_island.TestIsland.test_cell_move_herbivore_and_carnivore}}\pysiglinewithargsret{\sphinxbfcode{test\_cell\_move\_herbivore\_and\_carnivore}}{}{}
test for moving herbivore and carnivore
Returns
-------

\end{fulllineitems}

\index{test\_get\_direction() (biosim.tests.test\_island.TestIsland method)}

\begin{fulllineitems}
\phantomsection\label{\detokenize{tests:biosim.tests.test_island.TestIsland.test_get_direction}}\pysiglinewithargsret{\sphinxbfcode{test\_get\_direction}}{}{}
test for getting direction to move
Returns
-------

\end{fulllineitems}

\index{test\_get\_pi\_values\_carnivores\_no\_herb() (biosim.tests.test\_island.TestIsland method)}

\begin{fulllineitems}
\phantomsection\label{\detokenize{tests:biosim.tests.test_island.TestIsland.test_get_pi_values_carnivores_no_herb}}\pysiglinewithargsret{\sphinxbfcode{test\_get\_pi\_values\_carnivores\_no\_herb}}{}{}
test for getting pi\_values for carnivores without any herbivores
Returns
-------

\end{fulllineitems}

\index{test\_get\_pi\_values\_carnivores\_with\_herbs\_and\_carns() (biosim.tests.test\_island.TestIsland method)}

\begin{fulllineitems}
\phantomsection\label{\detokenize{tests:biosim.tests.test_island.TestIsland.test_get_pi_values_carnivores_with_herbs_and_carns}}\pysiglinewithargsret{\sphinxbfcode{test\_get\_pi\_values\_carnivores\_with\_herbs\_and\_carns}}{}{}
test for getting pi values

We also test with extra herbivores and carnivores in adjacent cells
Returns
-------

\end{fulllineitems}

\index{test\_get\_pi\_values\_herbivores() (biosim.tests.test\_island.TestIsland method)}

\begin{fulllineitems}
\phantomsection\label{\detokenize{tests:biosim.tests.test_island.TestIsland.test_get_pi_values_herbivores}}\pysiglinewithargsret{\sphinxbfcode{test\_get\_pi\_values\_herbivores}}{}{}
test for getting pi-values for herbivores
Returns
-------

\end{fulllineitems}

\index{test\_get\_random\_coordinates() (biosim.tests.test\_island.TestIsland method)}

\begin{fulllineitems}
\phantomsection\label{\detokenize{tests:biosim.tests.test_island.TestIsland.test_get_random_coordinates}}\pysiglinewithargsret{\sphinxbfcode{test\_get\_random\_coordinates}}{}{}
test for getting random coordinates, for type and length
Returns
-------

\end{fulllineitems}

\index{test\_migration() (biosim.tests.test\_island.TestIsland method)}

\begin{fulllineitems}
\phantomsection\label{\detokenize{tests:biosim.tests.test_island.TestIsland.test_migration}}\pysiglinewithargsret{\sphinxbfcode{test\_migration}}{}{}
test for migration
Returns
-------

\end{fulllineitems}

\index{test\_string\_to\_array() (biosim.tests.test\_island.TestIsland method)}

\begin{fulllineitems}
\phantomsection\label{\detokenize{tests:biosim.tests.test_island.TestIsland.test_string_to_array}}\pysiglinewithargsret{\sphinxbfcode{test\_string\_to\_array}}{}{}
Test for the method Island.string\_to\_array
Returns
-------

\end{fulllineitems}


\end{fulllineitems}



\section{The landscape test module}
\label{\detokenize{tests:the-landscape-test-module}}\phantomsection\label{\detokenize{tests:module-biosim.tests.test_landscape}}\index{biosim.tests.test\_landscape (module)}\index{TestLandscape (class in biosim.tests.test\_landscape)}

\begin{fulllineitems}
\phantomsection\label{\detokenize{tests:biosim.tests.test_landscape.TestLandscape}}\pysigline{\sphinxstrong{class }\sphinxcode{biosim.tests.test\_landscape.}\sphinxbfcode{TestLandscape}}
Class for testing landscape
\index{test\_add\_carnivores() (biosim.tests.test\_landscape.TestLandscape method)}

\begin{fulllineitems}
\phantomsection\label{\detokenize{tests:biosim.tests.test_landscape.TestLandscape.test_add_carnivores}}\pysiglinewithargsret{\sphinxbfcode{test\_add\_carnivores}}{}{}
method for testing add carnivore
Returns
-------

\end{fulllineitems}

\index{test\_add\_herbivore() (biosim.tests.test\_landscape.TestLandscape method)}

\begin{fulllineitems}
\phantomsection\label{\detokenize{tests:biosim.tests.test_landscape.TestLandscape.test_add_herbivore}}\pysiglinewithargsret{\sphinxbfcode{test\_add\_herbivore}}{}{}
method for testing add\_herbivore
Returns
-------

\end{fulllineitems}

\index{test\_aging() (biosim.tests.test\_landscape.TestLandscape method)}

\begin{fulllineitems}
\phantomsection\label{\detokenize{tests:biosim.tests.test_landscape.TestLandscape.test_aging}}\pysiglinewithargsret{\sphinxbfcode{test\_aging}}{}{}
Test that all animals in the cell age: the method aging
Returns
-------

\end{fulllineitems}

\index{test\_death() (biosim.tests.test\_landscape.TestLandscape method)}

\begin{fulllineitems}
\phantomsection\label{\detokenize{tests:biosim.tests.test_landscape.TestLandscape.test_death}}\pysiglinewithargsret{\sphinxbfcode{test\_death}}{}{}
Test that some animals in the cell die: the method death

do this by maniplating omega
Returns
-------

\end{fulllineitems}

\index{test\_feeding\_carnivores() (biosim.tests.test\_landscape.TestLandscape method)}

\begin{fulllineitems}
\phantomsection\label{\detokenize{tests:biosim.tests.test_landscape.TestLandscape.test_feeding_carnivores}}\pysiglinewithargsret{\sphinxbfcode{test\_feeding\_carnivores}}{}{}
Test that all carnivores in the cell feeds: the method feeding

do this by manipulating parameter DeltaPhiMax and the food
Returns
-------

\end{fulllineitems}

\index{test\_feeding\_jungle() (biosim.tests.test\_landscape.TestLandscape method)}

\begin{fulllineitems}
\phantomsection\label{\detokenize{tests:biosim.tests.test_landscape.TestLandscape.test_feeding_jungle}}\pysiglinewithargsret{\sphinxbfcode{test\_feeding\_jungle}}{}{}
Test that all herbivores in the cell feed: the method feeding
Returns
-------

\end{fulllineitems}

\index{test\_feeding\_savannah() (biosim.tests.test\_landscape.TestLandscape method)}

\begin{fulllineitems}
\phantomsection\label{\detokenize{tests:biosim.tests.test_landscape.TestLandscape.test_feeding_savannah}}\pysiglinewithargsret{\sphinxbfcode{test\_feeding\_savannah}}{}{}
Test that all herbivores in the cell feed: the method feeding
Returns
-------

\end{fulllineitems}

\index{test\_get\_fodder() (biosim.tests.test\_landscape.TestLandscape method)}

\begin{fulllineitems}
\phantomsection\label{\detokenize{tests:biosim.tests.test_landscape.TestLandscape.test_get_fodder}}\pysiglinewithargsret{\sphinxbfcode{test\_get\_fodder}}{}{}
Test for method get\_fodder
Returns
-------

\end{fulllineitems}

\index{test\_loss\_of\_weight() (biosim.tests.test\_landscape.TestLandscape method)}

\begin{fulllineitems}
\phantomsection\label{\detokenize{tests:biosim.tests.test_landscape.TestLandscape.test_loss_of_weight}}\pysiglinewithargsret{\sphinxbfcode{test\_loss\_of\_weight}}{}{}
Test that all animal in cell lose weight:the method loss\_of\_weight
Returns
-------

\end{fulllineitems}

\index{test\_procreation() (biosim.tests.test\_landscape.TestLandscape method)}

\begin{fulllineitems}
\phantomsection\label{\detokenize{tests:biosim.tests.test_landscape.TestLandscape.test_procreation}}\pysiglinewithargsret{\sphinxbfcode{test\_procreation}}{}{}
Test that all animals in cell procreate: the method procreation

Do this by manipulating the parameter gamma
Returns
-------

\end{fulllineitems}

\index{test\_reduce\_fodder() (biosim.tests.test\_landscape.TestLandscape method)}

\begin{fulllineitems}
\phantomsection\label{\detokenize{tests:biosim.tests.test_landscape.TestLandscape.test_reduce_fodder}}\pysiglinewithargsret{\sphinxbfcode{test\_reduce\_fodder}}{}{}
Test for method reduce\_fodder
Returns
-------

\end{fulllineitems}

\index{test\_reset\_fodder\_jungle() (biosim.tests.test\_landscape.TestLandscape method)}

\begin{fulllineitems}
\phantomsection\label{\detokenize{tests:biosim.tests.test_landscape.TestLandscape.test_reset_fodder_jungle}}\pysiglinewithargsret{\sphinxbfcode{test\_reset\_fodder\_jungle}}{}{}
test for reset fodder in Jungle
Returns
-------

\end{fulllineitems}

\index{test\_reset\_fodder\_savannah() (biosim.tests.test\_landscape.TestLandscape method)}

\begin{fulllineitems}
\phantomsection\label{\detokenize{tests:biosim.tests.test_landscape.TestLandscape.test_reset_fodder_savannah}}\pysiglinewithargsret{\sphinxbfcode{test\_reset\_fodder\_savannah}}{}{}
Test for the method reset\_fodder i Savannah
Returns
-------

\end{fulllineitems}


\end{fulllineitems}



\section{The simulation test module}
\label{\detokenize{tests:the-simulation-test-module}}\phantomsection\label{\detokenize{tests:module-biosim.tests.test_simulation}}\index{biosim.tests.test\_simulation (module)}\index{TestSimulation (class in biosim.tests.test\_simulation)}

\begin{fulllineitems}
\phantomsection\label{\detokenize{tests:biosim.tests.test_simulation.TestSimulation}}\pysigline{\sphinxstrong{class }\sphinxcode{biosim.tests.test\_simulation.}\sphinxbfcode{TestSimulation}}
class for testing Simulation
\index{test\_add\_population() (biosim.tests.test\_simulation.TestSimulation method)}

\begin{fulllineitems}
\phantomsection\label{\detokenize{tests:biosim.tests.test_simulation.TestSimulation.test_add_population}}\pysiglinewithargsret{\sphinxbfcode{test\_add\_population}}{}{}
Tests the population is added to cell on island
Returns
-------

\end{fulllineitems}


\end{fulllineitems}



\chapter{Indices and tables}
\label{\detokenize{index:indices-and-tables}}\begin{itemize}
\item {} 
\DUrole{xref,std,std-ref}{genindex}

\item {} 
\DUrole{xref,std,std-ref}{modindex}

\item {} 
\DUrole{xref,std,std-ref}{search}

\end{itemize}


\renewcommand{\indexname}{Python Module Index}
\begin{sphinxtheindex}
\def\bigletter#1{{\Large\sffamily#1}\nopagebreak\vspace{1mm}}
\bigletter{b}
\item {\sphinxstyleindexentry{biosim.animals}}\sphinxstyleindexpageref{animals:\detokenize{module-biosim.animals}}
\item {\sphinxstyleindexentry{biosim.island}}\sphinxstyleindexpageref{island:\detokenize{module-biosim.island}}
\item {\sphinxstyleindexentry{biosim.landscape}}\sphinxstyleindexpageref{landscape:\detokenize{module-biosim.landscape}}
\item {\sphinxstyleindexentry{biosim.simulation}}\sphinxstyleindexpageref{simulation:\detokenize{module-biosim.simulation}}
\item {\sphinxstyleindexentry{biosim.tests.test\_animals}}\sphinxstyleindexpageref{tests:\detokenize{module-biosim.tests.test_animals}}
\item {\sphinxstyleindexentry{biosim.tests.test\_island}}\sphinxstyleindexpageref{tests:\detokenize{module-biosim.tests.test_island}}
\item {\sphinxstyleindexentry{biosim.tests.test\_landscape}}\sphinxstyleindexpageref{tests:\detokenize{module-biosim.tests.test_landscape}}
\item {\sphinxstyleindexentry{biosim.tests.test\_simulation}}\sphinxstyleindexpageref{tests:\detokenize{module-biosim.tests.test_simulation}}
\end{sphinxtheindex}

\renewcommand{\indexname}{Index}
\printindex
\end{document}